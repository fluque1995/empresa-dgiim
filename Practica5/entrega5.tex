\documentclass[11pt]{article}
\renewcommand{\baselinestretch}{1.05}
\usepackage[spanish]{babel}
\usepackage[utf8]{inputenc}
\usepackage{lipsum}

\usepackage{amsmath,amsthm,verbatim,amssymb,amsfonts,amscd}
\usepackage{graphicx, wrapfig}
\usepackage{float}
\usepackage{caption, subcaption}
\usepackage{tkz-fct}
\usetikzlibrary{babel}
\usepackage{pgfplots}
\usepackage{enumitem}
\usepackage{multicol, vwcol}
\usepackage{listingsutf8}
\usepackage{color}
\usepackage{hyperref}
\usepackage{booktabs}
\usepackage{ marvosym }
\definecolor{lightgray}{gray}{0.95}

\hypersetup{
    bookmarks=true,         % show bookmarks bar?
    unicode=false,          % non-Latin characters in Acrobat’s bookmarks
    pdftoolbar=true,        % show Acrobat’s toolbar?
    pdfmenubar=true,        % show Acrobat’s menu?
    pdffitwindow=false,     % window fit to page when opened
    pdfstartview={FitW},    % fits the width of the page to the window
    pdftitle={Ingeniería, empresa y sociedad - Entrega 5},    % title
    pdfauthor={Francisco Luque, María del Mar Ruiz, Miguel Morales},     % author
    pdfsubject={Ingeniería, Empresa y Sociedad},   % subject of the document
    pdfnewwindow=true,      % links in new PDF window
    colorlinks=true,        % false: boxed links; true: colored links
    linkcolor=red,          % color of internal links (change box color with linkbordercolor)
    citecolor=cyan,         % color of links to bibliography
    filecolor=magenta,      % color of file links
    urlcolor=blue           % color of external links
}

\topmargin0.0cm
\headheight0.0cm
\headsep0.0cm
\oddsidemargin0.0cm
\textheight23.0cm
\textwidth16.5cm
\footskip1.0cm
\theoremstyle{plain}

\newtheorem{theorem}{Teorema}
\newtheorem{corollary}{Corolario}
\newtheorem{lemma}{Lema}
\newtheorem{proposition}{Proposición}
\theoremstyle{definition}
\newtheorem{definition}{Definición}
\newtheorem{example}{Ejemplo}

\newcommand{\N}{\mathbb{N}}
\newcommand{\Z}{\mathbb{Z}}
\newcommand{\Q}{\mathbb{Q}}
\newcommand{\C}{\mathbb{C}}
\newcommand{\R}{\mathbb{R}}

\begin{document}

\title{Ingeniería, Empresa y Sociedad \\
  DGIIM \\
  \large La estrategia competitiva de Naranjas del Carmen}
\author{\textit{1KB-RAM}\\
  Francisco Luque Sánchez\\
  Miguel Morales Castillo\\
  María del Mar Ruiz Martín}

\maketitle

\textbf{Ejercicio:} Vamos a diferenciar la estrategia competitiva de
la empresa en dos segmentos de mercado distintos. Por un lado, vamos
a estudiar la estrategia competitiva de la empresa en el mercado
nacional, y por otro la estrategia de la empresa en el mercado
internacional. Comenzamos con la estrategia a nivel nacional. En este
caso, la estrategia de la empresa es claramente una diferenciación del
producto. Ofrecen un producto (naranjas) de mucha calidad (puedes
recibir el producto en casa un día después de su recolección directa
del árbol), lo que les permite tener unos precios ligeramente más
altos que su competencia. La fuente de esa diferenciación, como ya
hemos dicho, reside en unas naranjas de calidad, lo que les da la
ventaja de tener unos precios ligeramente más altos que la
competencia.  No obstante, este producto de tanta calidad tiene
también el inconveniente de que la producción no puede ser muy grande,
lo que restringe ligeramente la capacidad de generar ingresos para
la empresa.\\

En el mercado internacional, por otra parte, esta ventaja de
diferenciación del producto se une a una ventaja de liderazgo en
costes. El precio de la fruta en países de la unión europea es
usualmente muy alto, lo que hace que el precio al que vende esta
empresa no sea más caro que la oferta que ya hay.\\

\end{document}
