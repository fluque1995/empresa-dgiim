\documentclass[11pt]{article}
\renewcommand{\baselinestretch}{1.05}
\usepackage[spanish]{babel}
\usepackage[utf8]{inputenc}
\usepackage{lipsum}

\usepackage{amsmath,amsthm,verbatim,amssymb,amsfonts,amscd}
\usepackage{graphicx, wrapfig}
\usepackage{float}
\usepackage{caption, subcaption}
\usepackage{tkz-fct}
\usetikzlibrary{babel}
\usepackage{pgfplots}
\usepackage{enumitem}
\usepackage{multicol, vwcol}
\usepackage{listingsutf8}
\usepackage{color}
\usepackage{hyperref}
\usepackage{booktabs}
\usepackage{ marvosym }
\usepackage[sorting=none]{biblatex}
\bibliography{bibliography.bib}
\definecolor{lightgray}{gray}{0.95}

\hypersetup{
    bookmarks=true,         % show bookmarks bar?
    unicode=false,          % non-Latin characters in Acrobat’s bookmarks
    pdftoolbar=true,        % show Acrobat’s toolbar?
    pdfmenubar=true,        % show Acrobat’s menu?
    pdffitwindow=false,     % window fit to page when opened
    pdfstartview={FitW},    % fits the width of the page to the window
    pdftitle={Ingeniería, empresa y sociedad - Entrega 2},    % title
    pdfauthor={Francisco Luque, María del Mar Ruiz},     % author
    pdfsubject={Ingeniería, Empresa y Sociedad},   % subject of the document
    pdfnewwindow=true,      % links in new PDF window
    colorlinks=true,        % false: boxed links; true: colored links
    linkcolor=red,          % color of internal links (change box color with linkbordercolor)
    citecolor=cyan,         % color of links to bibliography
    filecolor=magenta,      % color of file links
    urlcolor=blue           % color of external links
}

\setlength{\parindent}{0pt}
\topmargin0.0cm
\headheight0.0cm
\headsep0.0cm
\oddsidemargin0.0cm
\textheight23.0cm
\textwidth16.5cm
\footskip1.0cm
\theoremstyle{plain}

\newtheorem{theorem}{Teorema}
\newtheorem{corollary}{Corolario}
\newtheorem{lemma}{Lema}
\newtheorem{proposition}{Proposición}
\theoremstyle{definition}
\newtheorem{definition}{Definición}
\newtheorem{example}{Ejemplo}

\newcommand{\N}{\mathbb{N}}
\newcommand{\Z}{\mathbb{Z}}
\newcommand{\Q}{\mathbb{Q}}
\newcommand{\C}{\mathbb{C}}
\newcommand{\R}{\mathbb{R}}

\begin{document}

\title{Ingeniería, Empresa y Sociedad \\
  DGIIM \\
  \large Estructura de propiedad, gobierno corporativo y
  responsabilidad social corporativa de Inditex}
\author{Francisco Luque Sánchez\\
  María del Mar Ruiz Martín}
\maketitle

En este caso práctico se desarrollará un informe sobre las
características de la empresa INDITEX. En primer lugar, se comentarán
algunas características generales de la misma, utilizando para ello
la información que se muestra en su página web, así como los informes
que nos han sido entregados. A continuación, comentaremos brevemente
su gobierno corporativo, la estructura de propiedad, y su responsabilidad
social corporativa.\\

Comenzamos describiendo algunas características de la empresa.

\section*{Apartado 1: Descripción de la empresa}

INDITEX (acrónimo de Industria de diseño textil) es un grupo
multinacional español fundado en el año 1985. Su sector de actividad
es la industria manufacturera, concretamente la industria
textil. Actualmente cuenta con más de 162000 empleados en todo el
mundo, repartidos en más de 7500 tiendas. A este número hay que
sumarle las más de 3700 fábricas
repartidas también por todo el territorio mundial.\\

En cuanto a los mercados en los que tiene localizada su actividad de
producción, cabe destacar Europa, donde tienen más de 5000 tiendas.  A
continuación, aparece américa, donde se encuentran aproximadamente
800 tiendas, teniendo situadas las demás por el resto del mundo.\\

Atendiendo al volumen de negocio, según los datos de la página web, en
2016 (último dato que se ofrece, no está disponible todavía el dato de
2017) se facturaron en ventas 23311 millones de euros. En cuanto al
beneficio bruto, ese año se consiguieron 5083 millones de euros, que
tras el pago de impuestos y tasas se convierten en 3157 millones.
Además, cabe destacar que desde el año 2008, año en el que comienzan
los registros que se muestran en la página web, la empresa ha conseguido
siempre un mayor nivel de ingresos que el año anterior. Concretamente,
desde el año 2008 se han duplicado las ganancias anuales de la empresa.

\section*{Apartado 2: Información sobre el gobierno corporativo en la
  página web}
\section*{Apartado 3: Estructura de propiedad de la empresa}
\section*{Apartado 4: Gobierno corporativo y responsabilidad social
  corporativa}

\end{document}
