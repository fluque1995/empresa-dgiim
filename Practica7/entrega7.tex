\documentclass[11pt]{article}
\renewcommand{\baselinestretch}{1.05}
\usepackage[spanish]{babel}
\usepackage[utf8]{inputenc}
\usepackage{lipsum}

\usepackage{amsmath,amsthm,verbatim,amssymb,amsfonts,amscd}
\usepackage{graphicx, wrapfig}
\usepackage{float}
\usepackage{caption, subcaption}
\usepackage{tkz-fct}
\usetikzlibrary{babel}
\usepackage{pgfplots}
\usepackage{enumitem}
\usepackage{multicol, vwcol}
\usepackage{listingsutf8}
\usepackage{color}
\usepackage{hyperref}
\usepackage{booktabs}
\usepackage{lmodern, textcomp}
\usepackage{ marvosym }
\definecolor{lightgray}{gray}{0.95}

\usepackage{booktabs}

\hypersetup{
    bookmarks=true,         % show bookmarks bar?
    unicode=false,          % non-Latin characters in Acrobat’s bookmarks
    pdftoolbar=true,        % show Acrobat’s toolbar?
    pdfmenubar=true,        % show Acrobat’s menu?
    pdffitwindow=false,     % window fit to page when opened
    pdfstartview={FitW},    % fits the width of the page to the window
    pdftitle={Ingeniería, empresa y sociedad - Entrega 7},    % title
    pdfauthor={Francisco Luque, María del Mar Ruiz, Miguel Morales},     % author
    pdfsubject={Ingeniería, Empresa y Sociedad},   % subject of the document
    pdfnewwindow=true,      % links in new PDF window
    colorlinks=true,        % false: boxed links; true: colored links
    linkcolor=red,          % color of internal links (change box color with linkbordercolor)
    citecolor=cyan,         % color of links to bibliography
    filecolor=magenta,      % color of file links
    urlcolor=blue           % color of external links
}

\topmargin0.0cm
\headheight0.0cm
\headsep0.0cm
\oddsidemargin0.0cm
\textheight23.0cm
\textwidth16.5cm
\footskip1.0cm
\theoremstyle{plain}

\newtheorem{theorem}{Teorema}
\newtheorem{corollary}{Corolario}
\newtheorem{lemma}{Lema}
\newtheorem{proposition}{Proposición}
\theoremstyle{definition}
\newtheorem{definition}{Definición}
\newtheorem{example}{Ejemplo}

\newcommand{\N}{\mathbb{N}}
\newcommand{\Z}{\mathbb{Z}}
\newcommand{\Q}{\mathbb{Q}}
\newcommand{\C}{\mathbb{C}}
\newcommand{\R}{\mathbb{R}}

\begin{document}

\title{Ingeniería, Empresa y Sociedad \\
  DGIIM \\
  \large La dirección financiera}
\author{\textit{1KB-RAM}\\
  Francisco Luque Sánchez\\
  Miguel Morales Castillo\\
  María del Mar Ruiz Martín}

\maketitle

\textbf{Ejercicio 1 a:} Elaborar el Balance de Situación de la empresa
y calcular el resultado de la empresa.\\

\begin{table}[H]
  \centering
  \begin{tabular}{lrlr}
    \toprule
    \multicolumn{2}{c}{Activo} & \multicolumn{2}{c}{Pasivo} \\
    \cmidrule(r){1-2} \cmidrule(l){3-4}
    \multicolumn{2}{c}{Activo no corriente} & \multicolumn{2}{c}{Patrimonio neto} \\
    \cmidrule(r){1-2} \cmidrule(l){3-4}
    \multicolumn{2}{c}{Inmovilizado intangible} & 109000 € & Capital social\\
    \cmidrule(r){1-2}
    10000 € & Aplicación informática & 30000 € & Reserva legal \\
    \cmidrule(r){1-2}
    \multicolumn{2}{c}{Inmovilizado material} & 550000 € & Resultado del ejercicio \\
    \cmidrule(r){1-2} \cmidrule(l){3-4}
    400000 € & Terreno & \multicolumn{2}{c}{Pasivo no corriente} \\
    \cmidrule(l){3-4}
    600000 € & Instalaciones & 680000 € & Préstamos L/P \\
    \cmidrule(l){3-4}
    60000 € & Equipos informáticos & \multicolumn{2}{c}{Pasivo corriente} \\
    \cmidrule(l){3-4}
    30000 € & Elementos de transporte & 16000 € & Letras a pagar\\
    12000 € & Mobiliario & 8000 € & Seguridad social \\
    -190000 € & Amortización inmovilizado & 11000 € & Hacienda pública \\
    \cmidrule(r){1-2}
    \multicolumn{2}{c}{Activo corriente} & 75000 € & Préstamo a C/P  \\
    \cmidrule(r){1-2}
    32000 € & Mercaderías & & \\
    400000 € & Clientes no documentados & & \\
    25000 € & Clientes documentados & & \\
    80000 € & Bancos & & \\
    20000 € & Caja & & \\
    \midrule
    1479000 € & Total activo & 1479000 € & Total pasivo \\
    \bottomrule
  \end{tabular}
  \caption{Balance de situación}
\end{table}

\textbf{Ejercicio 1 b:} Calcular y comentar el Fondo de Maniobra.\\

El fondo de maniobra de la empresa de define como el activo corriente
menos el pasivo corriente. En este caso, el fondo de maniobra vale
$557000 - 110000 = 447000$ €, lo cual es positivo para la empresa, ya
que reduce el riesgo de tener que afrontar pagos a corto plazo y no
tener liquidez para responder a los mismos.\\

\textbf{Ejercicio 1 c:} Los elementos de transporte corresponden a dos
furgonetas. Estimar, según las tablas de amortización, la cuota máxima
y la cuota mínima a emplear.\\

Tenemos dos furgonetas, a las que van destinados 15000 € para cada una.
La cuota máxima es por tanto $15000 * 0.2 = 3000$ € por furgoneta al
año. La mínima, por otro lado, es de $\frac{15000}{10} = 1500$ €.
El valor de la tabla con el que se ha trabajado es el de autocamiones.\\

\textbf{Ejercicio 2 a:} Calcular el beneficio antes de intereses e
impuestos (BAIT), el beneficio antes de impuestos (BAT) y el beneficio
neto (BN).\\

Comenzamos calculando el BAIT. Dicho valor se calcula como Ingresos -
Coste de los bienes - Gastos Operativos. Tenemos entonces:

\[
  (33 + 47 + 55 + 38) - (22 + 24 + 28 + 25) - 44 = 173 - 99 - 44 = 30
  \text{ millones}
\]

El BAT en este caso coincide con el BAIT, por lo que son también 30
millones.  El BN, dado que tenemos un tipo impositivo sobre los
beneficios del $35 \%$, tenemos que es $30 * 0.65 = 19.50$ millones.\\

\textbf{Ejercicio 2 b:} Calcular los $Q_j$ y el FNC total.\\

Pasamos a calcular los $Q_j$:

\[
  Q_1 = 33 - 22 = 11 \text{ millones}
\]
\[
  Q_2 = 23 \text{ millones}
\]
\[
  Q_3 = 27 \text{ millones}
\]
\[
  Q_4 = 13 \text{ millones}
\]

Y el FNCT:

\[
  FNCT = -A_0 + \sum Q_j = -44 + 74 = 30\text{ millones}
\]

\textbf{Ejercicio 2 c:} Calcular el VAN y la TIR. Comentar los
resultados.\\

Calculamos primero el VAN:

\[
  VAN = -A_0 + \frac{Q_1}{1.1} + \frac{Q_2}{1.1^2} + \frac{Q_3}{1.1^3}
  + \frac{Q_4}{1.1^4} = -44 + 10 + 19.008 + 20.285 + 8.879 = 14.172
\]

Y calculamos ahora el TIR, que utilizando la funcionalidad de excel
que nos lo permite nos arroja un valor de $r = 23.39 \%$.\\

\textbf{Ejercicio 3 a:} Calcular el resultado del ejercicio y el
capital social de la empresa.\\

Vamos a calcular primeramente el resultado del ejercicio. Para ello,
lo primero es calcular los gatos que ha tenido la empresa. Los gastos
que tenemos son los siguientes:

\begin{table}[H]
  \centering
  \begin{tabular}{lr}
    Intereses (5\% de 52000 €) & 2600 €\\
    Gastos fijos & 35000 €\\
    Gastos variables & 15000 €\\
    Amortización del inmovilizado intangible & 3000 €\\
    Amortización del inmovilizado material & 500 €\\
    \midrule
    Total gastos & 56100 €
  \end{tabular}
  \caption{Gastos generados por la empresa}
\end{table}

El resultado del ejercicio es, entonces, la diferencia entre los
ingresos y los gastos generados por la empresa. En este caso,
ascienden a $65000 - 56100 = 8900 €$.\\

De estos 8900 €, hay que destinar un 25 \% a impuestos, lo cual
asciende a 2225 €. El capital social de la empresa nos aparecerá en el
siguiente ejercicio, cuando elaboremos el balance, sabiendo que el
total activo y total pasivo tienen que ser iguales. Adelantamos que el
valor del mismo será de 275 €.\\

\textbf{Ejercicio 3 b:} Elaborar el Balance de Situación de la
empresa.\\

\begin{table}[H]
  \centering
  \begin{tabular}{lrlr}
    \toprule
    \multicolumn{2}{c}{Activo} & \multicolumn{2}{c}{Pasivo} \\
    \cmidrule(r){1-2} \cmidrule(l){3-4}
    \multicolumn{2}{c}{Activo no corriente} & \multicolumn{2}{c}{Patrimonio neto} \\
    \cmidrule(r){1-2} \cmidrule(l){3-4}
    \multicolumn{2}{c}{Inmovilizado intangible} & 275 € & Capital social\\
    \cmidrule(r){1-2}
    2000 € & Aplicaciones informática & 500 € & Reserva legal \\
    500 € & Am. Acum. Inm. Intangible & 8900 € & Resultado del ejercicio \\
    \cmidrule(r){1-2}
    \multicolumn{2}{c}{Inmovilizado material} & -8000 € & Resultado del ejercicio anterior \\
    \cmidrule(r){1-2} \cmidrule(l){3-4}
    3000 € & Equipos informáticos & \multicolumn{2}{c}{Pasivo no corriente} \\
    \cmidrule(l){3-4}
    10000 € & Elementos de transporte & 52000 € & Deudas L/P \\
    \cmidrule(l){3-4}
    5000 € & Mobiliario & \multicolumn{2}{c}{Pasivo corriente} \\
    \cmidrule(l){3-4}
    -3000 € & Am. Acum. Inm. Material & 3000 € & Acreedores \\
    \cmidrule(r){1-2}
    \multicolumn{2}{c}{Activo corriente} & 2500 € & Proveedores \\
    \cmidrule(r){1-2}
    6000 € & Existencias & 11000 € & Hacienda pública \\
    2500 € & Clientes & & \\
    3000 € & Bancos L/P & & \\
    35675 € & Caja & & \\
    \midrule
    65275 € & Total activo & 65275 € & Total pasivo \\
    \bottomrule
  \end{tabular}
  \caption{Balance de situación}
\end{table}

\textbf{Ejercicio 3 c:} Calcular e interpretar el fondo de maniobra.\\

El fondo de maniobra de la empresa, como ya dijimos antes, es el
activo corriente menos el pasivo corriente, valor que en este caso
asciende a $45675 - 5500 = 40175$ €, lo cual es algo muy positivo para
la empresa, ya que le hace tener un riesgo muy bajo a la hora de tener
que enfrentar deudas a corto plazo.

\end{document}
