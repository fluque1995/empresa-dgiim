\documentclass[11pt]{article}
\renewcommand{\baselinestretch}{1.05}
\usepackage[spanish]{babel}
\usepackage[utf8]{inputenc}
\usepackage{lipsum}

\usepackage{amsmath,amsthm,verbatim,amssymb,amsfonts,amscd}
\usepackage{graphicx, wrapfig}
\usepackage{float}
\usepackage{caption, subcaption}
\usepackage{tkz-fct}
\usetikzlibrary{babel}
\usepackage{pgfplots}
\usepackage{enumitem}
\usepackage{multicol, vwcol}
\usepackage{listingsutf8}
\usepackage{color}
\usepackage{hyperref}
\usepackage{booktabs}
\usepackage{ marvosym }
\usepackage[sorting=none]{biblatex}
\bibliography{bibliography.bib}
\definecolor{lightgray}{gray}{0.95}

\hypersetup{
    bookmarks=true,         % show bookmarks bar?
    unicode=false,          % non-Latin characters in Acrobat’s bookmarks
    pdftoolbar=true,        % show Acrobat’s toolbar?
    pdfmenubar=true,        % show Acrobat’s menu?
    pdffitwindow=false,     % window fit to page when opened
    pdfstartview={FitW},    % fits the width of the page to the window
    pdftitle={Ingeniería, empresa y sociedad - Entrega 2},    % title
    pdfauthor={Francisco Luque, María del Mar Ruiz},     % author
    pdfsubject={Ingeniería, Empresa y Sociedad},   % subject of the document
    pdfnewwindow=true,      % links in new PDF window
    colorlinks=true,        % false: boxed links; true: colored links
    linkcolor=red,          % color of internal links (change box color with linkbordercolor)
    citecolor=cyan,         % color of links to bibliography
    filecolor=magenta,      % color of file links
    urlcolor=blue           % color of external links
}

\topmargin0.0cm
\headheight0.0cm
\headsep0.0cm
\oddsidemargin0.0cm
\textheight23.0cm
\textwidth16.5cm
\footskip1.0cm
\theoremstyle{plain}

\newtheorem{theorem}{Teorema}
\newtheorem{corollary}{Corolario}
\newtheorem{lemma}{Lema}
\newtheorem{proposition}{Proposición}
\theoremstyle{definition}
\newtheorem{definition}{Definición}
\newtheorem{example}{Ejemplo}

\newcommand{\N}{\mathbb{N}}
\newcommand{\Z}{\mathbb{Z}}
\newcommand{\Q}{\mathbb{Q}}
\newcommand{\C}{\mathbb{C}}
\newcommand{\R}{\mathbb{R}}

\begin{document}

\title{Ingeniería, Empresa y Sociedad \\
  DGIIM \\
  \large El entorno de la empresa: Llamadas por VoIP contra la
  telefonía convencional }
\author{Francisco Luque Sánchez\\
  Miguel Morales Castillo María del Mar Ruiz Martín}

\maketitle

\textbf{Ejercicio 1:} En el modelo de las 5 fuerzas competitivas de
Porter, ¿qué representaría el nuevo servicio de llamadas de Whatsapp
para las empresas de telecomunicación en 2015? ¿Crees que suponía en
aquel entonces una amenaza relevante?\\

La aparición de las llamadas de Whatsapp supondrían la aparición de
un producto sustitutivo, puesto que cubre la misma necesidad, esto es,
de comunicación mediante llamadas, a partir de una nueva tecnología.\\

En cuanto a si suponía una amenaza relevante en aquel momento, podemos
razonar de dos formas distintas. Si tenemos en cuenta el momento en el
que surgieron las llamadas de voz de Whatsapp, en el que la tecnología
no estaba completamente desarrollada, y el cambio de una forma de
funcionamiento a otra (el cambiar de llamar desde el teléfono a llamar
desde otra aplicación) suele suponer una barrera para los usuarios,
especialmente cuando hablamos de temas de tecnología. Esto hace que la
amenaza no fuera relevante en aquel momento. No obstante, la base de
usuarios tan grande que tiene Whatsapp, unido a la posibilidad de una
mejora en la tecnología que la convierta en una herramienta viable, sí
que se podía considerar una amenaza relevante en un futuro.\\

\textbf{Ejercicio 2:} Ante el lanzamiento en 2015 del nuevo servicio
de llamadas de Whatsapp, ¿qué reacciones se esperaban por parte de las
compañías de telefonía móvil?\\


Principalmente se esperaban dos posibles respuestas: aumentar los 
minutos en las tarifas de llamadas o aprovecharse del posible aute
de la VoIP para promocionarse ya que sus datos podrían usarse para
dichas llamadas.\\

También, aunque no lo consideraban factible, comentan la posibilidad
de disminuir el servicio de las VoIP. De esta forma reducirían el
número de usuarios que migraran al nuevo servicio.\\



\textbf{Ejercicio 3:} Según el informe de Cisco Systems citado en el
artículo, ``se prevé que en 2018 los minutos de llamadas por WiFi
superen los minutos de llamadas por la red telefónica
convencional''. Ante la cercanía del momento pronosticado, ¿crees que
se han cumplido las previsiones? ¿Cómo calificarías, según los
parámetros estudiados en clase, la amenaza de este servicio para las
operadoras de telefonía tradicionales?\\

\printbibliography

\end{document}
