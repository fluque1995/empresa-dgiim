\documentclass[11pt]{article}
\renewcommand{\baselinestretch}{1.05}
\usepackage[spanish]{babel}
\usepackage[utf8]{inputenc}
\usepackage{lipsum}

\usepackage{amsmath,amsthm,verbatim,amssymb,amsfonts,amscd}
\usepackage{graphicx, wrapfig}
\usepackage{float}
\usepackage{caption, subcaption}
\usepackage{tkz-fct}
\usetikzlibrary{babel}
\usepackage{pgfplots}
\usepackage{enumitem}
\usepackage{multicol, vwcol}
\usepackage{listingsutf8}
\usepackage{color}
\usepackage{hyperref}
\usepackage{booktabs}
\usepackage{ marvosym }
\definecolor{lightgray}{gray}{0.95}

\hypersetup{
    bookmarks=true,         % show bookmarks bar?
    unicode=false,          % non-Latin characters in Acrobat’s bookmarks
    pdftoolbar=true,        % show Acrobat’s toolbar?
    pdfmenubar=true,        % show Acrobat’s menu?
    pdffitwindow=false,     % window fit to page when opened
    pdfstartview={FitW},    % fits the width of the page to the window
    pdftitle={Ingeniería, empresa y sociedad - Entrega 3},    % title
    pdfauthor={Francisco Luque, María del Mar Ruiz},     % author
    pdfsubject={Ingeniería, Empresa y Sociedad},   % subject of the document
    pdfnewwindow=true,      % links in new PDF window
    colorlinks=true,        % false: boxed links; true: colored links
    linkcolor=red,          % color of internal links (change box color with linkbordercolor)
    citecolor=cyan,         % color of links to bibliography
    filecolor=magenta,      % color of file links
    urlcolor=blue           % color of external links
}

\topmargin0.0cm
\headheight0.0cm
\headsep0.0cm
\oddsidemargin0.0cm
\textheight23.0cm
\textwidth16.5cm
\footskip1.0cm
\theoremstyle{plain}

\newtheorem{theorem}{Teorema}
\newtheorem{corollary}{Corolario}
\newtheorem{lemma}{Lema}
\newtheorem{proposition}{Proposición}
\theoremstyle{definition}
\newtheorem{definition}{Definición}
\newtheorem{example}{Ejemplo}

\newcommand{\N}{\mathbb{N}}
\newcommand{\Z}{\mathbb{Z}}
\newcommand{\Q}{\mathbb{Q}}
\newcommand{\C}{\mathbb{C}}
\newcommand{\R}{\mathbb{R}}

\begin{document}

\title{Ingeniería, Empresa y Sociedad \\
  DGIIM \\
  \large Análisis PESTEL: La burbuja inmobiliaria y el sector de la
  construcción}
\author{\textit{1KB-RAM}\\
  Francisco Luque Sánchez\\
  Miguel Morales Castillo \\
  María del Mar Ruiz Martín }

\maketitle

\textbf{Ejercicio 1:} Considerando como nivel geográfico de análisis
el territorio español, ¿qué factores del entorno general (PESTEL) que
afectaron al sector de la construcción y a su evolución en las décadas
de 1990 y 2010 identificas en el vídeo? Dicho de otra forma: imagina
que en aquellos años eras empresario de la construcción y del sector
inmobiliario; en ese caso, ¿qué factores habrías incluido en un
análisis PEST(EL), de los que se muestran en el vídeo?\\

\textbf{Ejercicio 1:}¿Qué cosas nuevas has aprendido viendo el vídeo?
Explica las ideas que más te han llamado la atención y/o consideras
más relevantes.

\end{document}
