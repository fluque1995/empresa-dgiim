\documentclass[11pt]{article}
\renewcommand{\baselinestretch}{1.05}
\usepackage[spanish]{babel}
\usepackage[utf8]{inputenc}
\usepackage{lipsum}

\usepackage{amsmath,amsthm,verbatim,amssymb,amsfonts,amscd}
\usepackage{graphicx, wrapfig}
\usepackage{float}
\usepackage{caption, subcaption}
\usepackage{tkz-fct}
\usetikzlibrary{babel}
\usepackage{pgfplots}
\usepackage{enumitem}
\usepackage{multicol, vwcol}
\usepackage{listingsutf8}
\usepackage{color}
\usepackage{hyperref}
\usepackage{booktabs}
\usepackage{ marvosym }
\definecolor{lightgray}{gray}{0.95}

\hypersetup{
    bookmarks=true,         % show bookmarks bar?
    unicode=false,          % non-Latin characters in Acrobat’s bookmarks
    pdftoolbar=true,        % show Acrobat’s toolbar?
    pdfmenubar=true,        % show Acrobat’s menu?
    pdffitwindow=false,     % window fit to page when opened
    pdfstartview={FitW},    % fits the width of the page to the window
    pdftitle={Ingeniería, empresa y sociedad - Entrega 3},    % title
    pdfauthor={Francisco Luque, María del Mar Ruiz},     % author
    pdfsubject={Ingeniería, Empresa y Sociedad},   % subject of the document
    pdfnewwindow=true,      % links in new PDF window
    colorlinks=true,        % false: boxed links; true: colored links
    linkcolor=red,          % color of internal links (change box color with linkbordercolor)
    citecolor=cyan,         % color of links to bibliography
    filecolor=magenta,      % color of file links
    urlcolor=blue           % color of external links
}

\topmargin0.0cm
\headheight0.0cm
\headsep0.0cm
\oddsidemargin0.0cm
\textheight23.0cm
\textwidth16.5cm
\footskip1.0cm
\theoremstyle{plain}

\newtheorem{theorem}{Teorema}
\newtheorem{corollary}{Corolario}
\newtheorem{lemma}{Lema}
\newtheorem{proposition}{Proposición}
\theoremstyle{definition}
\newtheorem{definition}{Definición}
\newtheorem{example}{Ejemplo}

\newcommand{\N}{\mathbb{N}}
\newcommand{\Z}{\mathbb{Z}}
\newcommand{\Q}{\mathbb{Q}}
\newcommand{\C}{\mathbb{C}}
\newcommand{\R}{\mathbb{R}}

\begin{document}

\title{Ingeniería, Empresa y Sociedad \\
  DGIIM \\
  \large Análisis PESTEL: La burbuja inmobiliaria y el sector de la
  construcción}
\author{\textit{1KB-RAM}\\
  Francisco Luque Sánchez\\
  Miguel Morales Castillo\\
  María del Mar Ruiz Martín}

\maketitle

\textbf{Ejercicio 1:} Considerando como nivel geográfico de análisis
el territorio español, ¿qué factores del entorno general (PESTEL) que
afectaron al sector de la construcción y a su evolución en las décadas
de 1990 y 2010 identificas en el vídeo? Dicho de otra forma: imagina
que en aquellos años eras empresario de la construcción y del sector
inmobiliario; en ese caso, ¿qué factores habrías incluido en un
análisis PEST(EL), de los que se muestran en el vídeo?\\

En el vídeo se mencionan dos factores fundamentales, que son la Ley de
suelo del año 1998, y la Reforma laboral del año 2002, ambas
propuestas del entonces Presidente del Gobierno José María
Aznar. Estos dos factores, ambos englobados en la dimensión legal,
supusieron una oportunidad de oro para los empresarios de la
construcción. El primero de los factores, la Ley de suelo, aumentó la
superficie de terreno urbanizable, y aumentó mucho la rentabilidad del
negocio de la construcción. Esto propició que las empresas
constructoras proliferasen y empezasen a construir
indiscriminadamente, lo que debería
traer una disminución en el precio de la vivienda.\\

El segundo factor, la reforma laboral, también fue una gran
oportunidad, pero en este caso para los empresarios en general. Se
recortaban los derechos laborales para hacer más atractiva la
contratación, lo cual hizo que las empresas contrataran más fácilmente
y se redujera el paro. Esto también provocó que muchos jóvenes 
abandonasen los estudios. \\

Esta reducción del paro trajo consigo un cambio en el modo de vida de
la sociedad española (podemos considerar este cambio como un factor
socio-cultural). Los españoles comenzaron a comprar viviendas de forma
indiscriminada, lo que hizo que la ley de suelo no tuviese el efecto
deseado. El precio de la vivienda, dada la gran demanda que había, se
disparó, aumentando por consiguiente el precio del suelo, lo cual hizo
que el precio de la vivienda aumentase aún más. En este momento,
todavía en el año 2002, lo que comenzó siendo una oportunidad para las
empresas de la construcción empezaba a convertirse en una amenaza. A
pesar de dicho peligro, las empresas constructoras siguieron
construyendo a un ritmo vertiginoso, y en 2005 se construían más
viviendas en España que en varios países europeos juntos. Además, el
precio de la vivienda seguía subiendo, cosa que no ocurría con los
sueldos de los trabajadores. No obstante, la sensación de bonanza
económica que tenían los ciudadanos fue una oportunidad para los
bancos, que empezaron a dar créditos con unas condiciones muy poco
restrictivas, lo que provocó que la gente comenzase a contratar
créditos para pagar nuevas viviendas. Esto, junto con el estancamiento 
de los sueldos anteriormente citado, podría considerarse factores 
económicos. \\

En 2007, la deuda que habían contraído tanto los ciudadanos como las
entidades financieras españolas era desorbitada. Estos factores económicos
suponían una clara amenaza que desembocarían en la más intensa de ellas: una
crisis financiera que tuvo lugar en Estados Unidos en el año 2008. Fue
una amenaza importante para las empresas constructoras y las entidades
financieras. Dejaron de darse créditos y los inversores dejaron de
comprar deuda, por lo que se congeló la venta de viviendas, que provocó
una caída en picado de los precios, y el negocio de la construcción
dejó de ser rentable.\\

Finalmente se podría destacar que durante todo este periodo, se produce 
una total despreocupación por la inversión en I+D, ya que se descarta
fomentar las mismas. Por tanto, las escasas ayudas en I+D podría considerarse
un factor tecnológico que supone una oportunidad, puesto que la mayoría del 
gasto público y ayudas estaban orientados al negocio de la construcción.\\



\textbf{Ejercicio 2:} ¿Qué cosas nuevas has aprendido viendo el vídeo?
Explica las ideas que más te han llamado la atención y/o consideras
más relevantes.\\

Como ideas relevantes, podemos destacar el hecho de que se aprovechase 
la ley del suelo de forma indiscriminada y sin control alguno. De esta 
forma se permitió que en España se construyesen más viviendas que en 
Francia, Alemania e Italia juntas.\\

Otra cosa que nos ha llamado la atención, ha sido el rechazo que provocó 
la idea de aprovechar ese superávit para invertir en I+D. Así, si algún día
dejábamos de contar con ese crecimiento, habríamos fomentado el desarrollo
del país y quizás el resultado no habría sido tan negativo.\\

Resulta remarcable el hecho de que en menos de 20 años el precio del suelo
aumentase en casi un 320\%, más aún teniendo en cuenta que los salarios se
encontraban congelados, siendo en 2005 el sueldo medio español menos de la
mitad del alemán.\\

Ha resultado especialmente destacable la ironía subsyacente en el hecho de 
que, con el pretexto de la búsqueda de una España con mejores condiciones de 
vida, se tomasen medidas como la reducción de los derechos de los trabajadores.\\

También podemos señalar como España, a pesar de que los sueldos se encontraban
congelados, seguían manteniendo el mismo nivel de vida gracias a los bancos
sin pensar en si podían permitírselo o no. Los bancos, por su parte, daban créditos
de forma descontrolada sin pensar en si la persona que recibía el préstamo sería
capaz de devolverlo.\\

\end{document}
