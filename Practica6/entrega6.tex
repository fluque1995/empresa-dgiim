\documentclass[11pt]{article}
\renewcommand{\baselinestretch}{1.05}
\usepackage[spanish]{babel}
\usepackage[utf8]{inputenc}
\usepackage{lipsum}

\usepackage{amsmath,amsthm,verbatim,amssymb,amsfonts,amscd}
\usepackage{graphicx, wrapfig}
\usepackage{float}
\usepackage{caption, subcaption}
\usepackage{tkz-fct}
\usetikzlibrary{babel}
\usepackage{pgfplots}
\usepackage{enumitem}
\usepackage{multicol, vwcol}
\usepackage{listingsutf8}
\usepackage{color}
\usepackage{hyperref}
\usepackage{booktabs}
\usepackage{ marvosym }
\definecolor{lightgray}{gray}{0.95}

\hypersetup{
    bookmarks=true,         % show bookmarks bar?
    unicode=false,          % non-Latin characters in Acrobat’s bookmarks
    pdftoolbar=true,        % show Acrobat’s toolbar?
    pdfmenubar=true,        % show Acrobat’s menu?
    pdffitwindow=false,     % window fit to page when opened
    pdfstartview={FitW},    % fits the width of the page to the window
    pdftitle={Ingeniería, empresa y sociedad - Entrega 6},    % title
    pdfauthor={Francisco Luque, María del Mar Ruiz, Miguel Morales},     % author
    pdfsubject={Ingeniería, Empresa y Sociedad},   % subject of the document
    pdfnewwindow=true,      % links in new PDF window
    colorlinks=true,        % false: boxed links; true: colored links
    linkcolor=red,          % color of internal links (change box color with linkbordercolor)
    citecolor=cyan,         % color of links to bibliography
    filecolor=magenta,      % color of file links
    urlcolor=blue           % color of external links
}

\topmargin0.0cm
\headheight0.0cm
\headsep0.0cm
\oddsidemargin0.0cm
\textheight23.0cm
\textwidth16.5cm
\footskip1.0cm
\theoremstyle{plain}

\newtheorem{theorem}{Teorema}
\newtheorem{corollary}{Corolario}
\newtheorem{lemma}{Lema}
\newtheorem{proposition}{Proposición}
\theoremstyle{definition}
\newtheorem{definition}{Definición}
\newtheorem{example}{Ejemplo}

\newcommand{\N}{\mathbb{N}}
\newcommand{\Z}{\mathbb{Z}}
\newcommand{\Q}{\mathbb{Q}}
\newcommand{\C}{\mathbb{C}}
\newcommand{\R}{\mathbb{R}}

\begin{document}

\title{Ingeniería, Empresa y Sociedad \\
  DGIIM \\
  \large Snapchat vs. Zuckerberg}
\author{\textit{1KB-RAM}\\
  Francisco Luque Sánchez\\
  Miguel Morales Castillo\\
  María del Mar Ruiz Martín}

\maketitle

\textbf{Ejercicio 1:}

La estrategia inicial planteada por Zuckerberg era comprar Snapchat.
Puesto que no fue posible efectuar la compra, dada la negativa de
Snapchat, incorporó la misma funcionalidad a su aplicación, la cual
ya contaba con una base de usuarios importante. Además, en esta
dirección, la intención de Zuckerberg es crear una plataforma que
agrupe todos los servicios relacionados con la realidad aumentada.\\


\textbf{Ejercicio 2:}

En el contexto de la industria de la realidad aumentada, Facebook
Stories supuso un competidor potencial para Snapchat, puesto que
ofreció el mismo producto y anteriormente no suponía una competencia.

\end{document}
